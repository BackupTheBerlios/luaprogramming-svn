\documentclass{luaprogramming-doc}
\makeindex
\newdynamicframe[odd]{0.38\textwidth}{0.25\textheight}{0.8\textwidth}{0pt}[oddlogo]
\newdynamicframe[even]{0.85\textwidth}{0.25\textheight}{0pt}{0pt}[evenlogo]
\begin{document}
\setdynamiccontents*{oddlogo}{\includegraphics[width=0.2\textwidth]{lua-logo}}
\setdynamiccontents*{evenlogo}{\includegraphics[width=0.2\textwidth]{lua-logo}}
\title{\color{white}\Huge\sffamily\bfseries Lua Programming}
\author{\color{luaorange}\LARGE\sffamily\bfseries Vafa Khalighi}
\date{\color{luaorange}\large\sffamily\bfseries Draft}
\ffswapoddeven*{main}
\dfswapoddeven*{chaphead}

\pagenumbering{alph}
\maketitle
\setstaticframe*{backleft}{pages=none}
\setstaticframe*{backright}{pages=none}
\noindent
Copyright 2009--2010 by Vafa Khalighi\\
Sydney, Australia\\
\href{http://developer.berlios.de/projects/luaprogramming}{\texttt{http://developer.berlios.de/projects/luaprogramming}}\\
\href{mailto:vafa@users.berlios.de}{\texttt{vafa@users.berlios.de}}

\medskip
\noindent
Permission is granted to copy, distribute and/or modify \emph{the documentation}
under the terms of the \textsc{gnu} Free Documentation License, Version 1.3
or any later version published by the Free Software Foundation;
with no Invariant Sections, no Front-Cover Texts, and no Back-Cover Texts.
A copy of the license is included in the section entitled \textsc{gnu}
Free Documentation License.

\medskip\noindent  
Permission is granted to copy, distribute and/or modify \emph{the
document class used to typeset this documentation} under the terms of the \textsc{gnu} Public License, Version 2 or any later version published by the Free Software Foundation.
A copy of the license is included in the section entitled \textsc{gnu}
Public License.

\medskip
\noindent
Permission is also granted to distribute and/or modify \emph{both
the documentation and the document class} under the conditions of the LaTeX
Project Public License, either version 1.3 of this license or (at
your option) any later version. A copy of the license is included in
the section entitled \LaTeX\ Project Public License.

\frontmatter
% swap frames back again
\ffswapoddeven*{main}
\dfswapoddeven*{chaphead}
\thumbtabindex
\tableofcontents
\setdynamicframe*{footer}{pages=all}

\clearpage
\mainmatter



\chapter{Introduction}
\chapdesc{This chapter provides a brief overview of the package, 
the package options.}
\enablethumbtabs
\pagenumbering{arabic}
\section{What is bidi?}
Bi-directional text is text containing text in both text directionalities, both right-to-left (RTL) and left-to-right (LTR). It generally involves text containing different types of alphabets, but may also refer to boustrophedon, which is changing text directionality in each row.

Some writing systems of the world, notably the Arabic (including variants such as Nasta'liq), Persian and Hebrew scripts, are written in a form known as right-to-left (RTL), in which writing begins at the right-hand side of a page and concludes at the left-hand side. This is different from the left-to-right (LTR) direction used by most languages in the world. When LTR text is mixed with RTL in the same paragraph, each type of text should be written in its own direction, which is known as bi-directional text. This can get rather complex when multiple levels of quotation are used.
\begin{definition}
\textbf{lua}\\
Lua 5.1.4  Copyright (C) 1994-2008 Lua.org, PUC-Rio\\
>
\end{definition}
Some writing systems of the world, notably the Arabic (including variants such as Nasta'liq), Persian and Hebrew scripts, are written in a form known as right-to-left (RTL), in which writing begins at the right-hand side of a page and concludes at the left-hand side. This is different from the left-to-right (LTR) direction used by most languages in the world. When LTR text is mixed with RTL in the same paragraph, each type of text should be written in its own direction, which is known as bi-directional text. This can get rather complex when multiple levels of quotation are used.
Some writing systems of the world, notably the Arabic (including variants such as Nasta'liq), Persian and Hebrew scripts, are written in a form known as right-to-left (RTL), in which writing begins at the right-hand side of a page and concludes at the left-hand side. This is different from the left-to-right (LTR) direction used by most languages in the world. When LTR text is mixed with RTL in the same paragraph, each type of text should be written in its own direction, which is known as bi-directional text. This can get rather complex when multiple levels of quotation are used.



\disablethumbtabs

\printindex


\clearpage
% have a backcover:
\setdynamicframe*{footer}{pages=none}
\setstaticframe*{lastbackleft,lastbackright}{pages=even}
\mbox{}\clearpage
\end{document}